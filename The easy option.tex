The simplest option involves the fewest changes to the NICE process. Indeed, it would involve doing pretty much what it does now, only with slightly different (and more transparent) reasoning. In this scenario NICE would explicitly ignore the problem of affordability. Its remit would cease to be the consideration of optimality on a national level and it would ignore the budget constraint. NICE's remit would become figuring out which health technologies are 'worth it'; i.e. would the public be willing to purchase a given technology with a given health benefit at a given cost. To some extent, therefore, NICE would become a threshold-setter. The threshold should be based on some definition of a social value of a QALY. This is the easy option for NICE as setting the threshold would be the only additional task to what they currently do. Its threshold might not change all that much\cite{Donaldson_2011}, or may be a little higher\cite{Bobinac_2012}.

However, even if NICE denies responsibility, clearly someone does need to take account of affordability. Given the events associated with sofosbuvir it seems that this could become the work of NHS England. NHS England could use a threshold based on the budget and current QALY-productivity in the NHS. One might expect NHS England to be in a better position to identify the local evidence necessary to determine appropriate thresholds\cite{Appleby_2009}, which would likely be much lower than NICE's\cite{claxton2013methods}. It would also be responsible for disinvestment decisions. Given the nationwide remit of NHS England, this would still prevent postcode lotteries. The implication here, of course, is that NICE and NHS England might use different thresholds. Any number of decision rules could be used to determine the result for technologies falling between the two. Maybe this is where considerations for innovation or non-health-related equity concerns belong. It seems probable to me that NICE's threshold would be higher than NHS England's, in which case NICE would effectively be advising increases in the health budget. This is something that I quite like the sound of.