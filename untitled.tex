NICE recently completed their appraisal of the hepatitis C drug sofosbuvir\cite{National_Institute_for_Health_and_Care_Excellence2015-cp}. However, as has been reported in the media, NHS England will not be complying with the guidance within the normal time period\cite{Boseley2015-zg}.

The cost of a 24 week course of sofosbuvir is almost £70,000. Around 160,000 people are chronically infected with the hepatitis C virus in England\cite{Public_Health_England2014-in}, so that adds up to a fair chunk of the NHS budget. Yet the drug does appear to be cost-effective. ICERs differ for different patient groups, but for most scenarios the ICER is below £30,000 per QALY. In the NICE documentation, a number of reasons are listed for NHS England's decision. But what they ultimately boil down to --- it seems --- is affordability.

The problem is that NICE doesn't account for affordability in its guidance. One need only consider that the threshold has remained unchanged for over a decade to see that this is true. How to solve this problem really depends on what we believe the job of NICE should be. Should it be NICE's job to consider what should and shouldn't be purchased within the existing health budget? Or, rather, should it be NICE's job simply to figure out what is `worth it' to society, regardless of affordability? This isn't the first time that an NHS organisation has appealed against a NICE decision in some way\cite{Wells_2007}. Surely, it won't be the last. These instances represent a failure in the system, not least on grounds of accountability for reasonableness\cite{Daniels_2000}. Here I'd like to suggest that NICE has 3 options for dealing with this problem; one easy, one hard and one harder.