\section{The hard option}
Personally, I believe that NICE's failure to justify their threshold(s) is quite a serious failing and undermines the enterprise. The hard option will involve them defining it properly, informed by current levels of QALY-productivity in the NHS. Thus properly adopting a position as a threshold-searcher\cite{Culyer_2007}, and doing the job prescribed to NHS England in the 'easy option'. NICE guidance would therefore be informed by the current health budget and affordability, and therefore must include guidance on disinvestment. The first stage of this work has already been done\cite{claxton2013methods}. The disinvestment guidance would be the hard part. This argument has already been much discussed, and seems to be what many economists support\cite{Claxton_2014,Raftery_2009,McCabe_2008}.

I don't find this argument entirely compelling, at least not as a solution to the affordability problem. To solve this issue NICE would need to regularly review the current threshold and revise it in light of current productivity and the prevailing health budget. It has no experience of doing this. I believe the task could be more effectively carried out by commissioning organisations (such as NHS England), who are in a better position to oversee the collection of the appropriate data and would have a public responsibility to do so. It might also be politically useful if decisions about affordability were made independently of decisions about value.